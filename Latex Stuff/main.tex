\documentclass[12pt]{report}

\usepackage{amsmath,amsthm,amsfonts,amssymb}
\usepackage{mathtools}
\usepackage{graphicx}
\usepackage{color} 
\usepackage[thinlines]{easytable}
\usepackage{fancyhdr}
\usepackage{enumerate}
\usepackage[shortlabels]{enumitem}
\usepackage[dvipsnames]{xcolor}

\usepackage[titles]{tocloft}

% Language setting
% Replace `english' with e.g. `spanish' to change the document language
\usepackage[english]{babel}

% Set page size and margins
% Replace `letterpaper' with `a4paper' for UK/EU standard size
\usepackage[letterpaper,top=2cm,bottom=2cm,left=3cm,right=3cm,marginparwidth=1.75cm]{geometry}


\usepackage[colorlinks=true, allcolors=blue]{hyperref}



\begin{document}

\begin{titlepage}
    \centering
    \vspace*{1in}
    {\LARGE\bfseries UBC PATHFINDER\par}
    \vspace{1.5cm}
    {\Large by\par}
    \vspace{0.5cm}
    {\Large Autumn Cheung, JJ Kim, Ege Gures, Oliver Shen\par}
    \vspace{1cm}
    {\large B.Sc. The University of British Columbia\par}
    \vfill
    A LINEAR PROGRAMMING OPTIMZATION PROJECT SUBMITTED FOR MATH441\\
    \vspace{1cm}

    MATH441\_V 201 2024W2\\
    Dr. Patrick Walls\\
    \vspace{1cm}
    THE FACULTY OF SCIENCE\\
    DEPARTMENT OF MATHEMATICS
    \vfill
    {\large February $10^{\text{th}}$, 2025}
    \vfill

\end{titlepage}


\chapter*{Abstract}
hello 
\addcontentsline{toc}{section}{Abstract}

\tableofcontents



\section{Introduction}
The UBC Vancouver campus covers over 400 acres of land 
Our group aims to use linear programming to optimize the path between classes. Subsequently, we want to find the best dining hall on campus that minimizes the distance from the first class to the dining hall and from the dining hall to the second call.

\section{Problems}
\subsection{Shortest Path Between Two Classes}
By definition, a flow network is a direct graph thatz consists of nodes and edges, each of which has a capacity and receives a flow. Although network flow problems are typically associated with combinatorial linear programming, they can also be formulated as linear programming problems. This is achieved by defining variables that represent the flow on each edge of the network, incorporating constraints to ensure that the inflow and outflow at each node remain balanced, and optimizing the total flow while adhering to the capacity limitations of each edge. For our question, the relevant variables are as follows: 
\begin{itemize}
    \item $N$ is the set of all the building, road intersection points and endpoints (the nodes of the graph)
    \item $E$ is the set of all edges from node $i$ to $j$.
    \item $b_k$ ( the amount of flow generated by node k) $=$
    $\begin{cases}
b_o=1, & \text{origin node} 
\\b_d=-1, & \text{destination node}
\\b_k=0,& \text{otherwise}
\end{cases} $
    
    \item $d_{i,j}$ is the distance from one node to another node (the weight of the edges)
    \item $x_{i,j}$ (the flow from node\ $i$  to\ $j$) $=$ $\begin{cases}
1, & \text{if the edge} \ i\ \text{to}\  j \ \text{is part of the shortest path} \\
0, & \text{otherwise}
\end{cases} $
\item $u_{i,j}=1$ is the capacity constraint from node $i$ to $j$
\end{itemize}
Based on the formulation of the shortest path problem in Mason and Roy (p. 109), the linear programming problem is formulated as: 
\begin{align*}
    \text{minimize} \ &\sum_{i\in N}\sum_{j\in N}d_{ij}x_{ij}
    \\ \text{subject to} \ &\sum_{i \in N}x_{oi} - \sum_{j\in N}x_{jo} = b_0 \ \ \ \forall i,j \in N
    \\  &\sum_{i \in N}x_{di} - \sum_{j\in N}x_{jd} = b_d \ \ \ \forall i,j \in N
    \\ \ &\sum_{i \in N}x_{ki} - \sum_{j\in N}x_{jk} = b_k \ \ \ \  \forall i,j,k\in N
    \\& 0\leq x_{ij}\leq u_{ij}
    \\& d_{ij} > 0
    \\ & x_{ij}\in {0,1}
\end{align*}
The motivation of this problem is represented by the objective function: minimize $\sum_{i\in N}\sum_{j\in N}d_{ij}x_{ij}$. The function sums up the distance of all the possible edges in the route and optimizes it to provide the shortest route. On the other hand, to uphold the goal of the shortest path, the first three constraints are used to maintain flow conservation — the idea that only supply and source nodes of a network flow can have an outgoing or ongoing flow. The first constraint ensures that the origin node generates one unit of flow by keeping a balance between the total outgoing flow from and the incoming flow to the origin. Secondly, for the route to end at the specified destination, the second constraint ensures that the destination node consumes one unit of flow by balancing the total incoming flow to the destination and the outgoing flow. Finally, all intermediate nodes are subject to the third constraint, to establish a continuation of flow throughout the path without accumulating or losing flow. Combined, the linear programming problem provides the shortest possible route from the origin to the destination.

 
\subsection{Best Dining Option Between Two classes}

Given the network graph of UBC buildings represented as nodes, with feasible paths between them as edges weighted by distance, we seek to solve another problem.

Given a starting node and an ending node, the objective is to find an optimal path that must pass through exactly one restaurant from a given set of restaurant nodes. Each restaurant will be assigned a \textbf{utility score}, which represents the total satisfaction experienced from consuming the respective restaurant commodities. 

The goal is to minimize travel distance while also maximizing satisfaction. The problem is thus formulated as a \textbf{multi-objective} optimization, where we balance the shortest path with choosing the most satisfying restaurant.

The linear programming problem is formulated as follows:\\

\textbf{1. Decision variables}
\begin{itemize}
    \item $x_{ij}$: Binary variable indicating if edge $(i,j)$ is included in the path:
    
    $$x_{ij} = \begin{cases}
        1, & \text{if edge } (i,j) \text{ is included in the path}\\
        0, & \text{otherwise}
    \end{cases}$$

    \item $y_r$: Binary variable indicating if restaurant $r$ is chosen:
    
    $$y_r = \begin{cases}
        1, & \text{if restaurant } r \text{ is chosen}\\
        0, & \text{otherwise}
    \end{cases}$$
\end{itemize}

\textbf{2. Objective functions}
\begin{itemize}
    \item Minimize total travel distance:
    $$\min \sum_{i,j \in E} d_{ij}x_{ij}$$
    \begin{itemize}
        \item $d_{ij}$: Distance (weight) of edge $(i,j)$
        \item $E$: The set of all edges in the graph
    \end{itemize}

    \item Maximize utility:
    $$\max \sum_{r \in R} u_r y_r$$
    \begin{itemize}
        \item $u_r$: The utility score of restaurant $r$
        \item $R$: Set of restaurant nodes
    \end{itemize}

    We can simply rewrite this as a minimization problem:
    $$\min -\sum_{r \in R} u_r y_r$$
\end{itemize}

\textbf{3. Constraints}
\begin{itemize}
    \item Balance equation / Flow conservation
    $$\sum_{j \in N} x_{ij} - \sum_{j \in N} x_{ji} = \begin{cases}
        1, & \text{if } i=A\\
        -1, & \text{if } i=B\\
        0, & \text{otherwise}
    \end{cases}$$
    \begin{itemize}
        \item $N$ is the set of all nodes
        \item The path starts at $A$ and ends at $B$
        \item Intermediate nodes in between adhere to the flow balance
    \end{itemize}

    \item Exactly one restaurant must be visited
    $$\sum_{r\in R} y_r =1$$
    \begin{itemize}
        \item Ensures that only one restaurant is chosen
    \end{itemize}

    \item Chosen restaurant must be on the path
    $$y_r \leq \sum_{j \in N} x_{rj}, \; \forall r\in R $$
    \begin{itemize}
        \item If a restaurant $r$ is chosen, at least one edge leading to/from $r$ must be used
    \end{itemize}

    \item One meeting node is chosen
    
    

    \item Binary conditions on decision variables
    $$x_{ij} \in \{0,1\}, \; \forall (i,j) \in E$$
    $$y_r \in \{0,1\}, \; \forall r \in R$$
\end{itemize}


\textbf{THE IDEA HERE IS WE TURN A MULTI-OBJECTIVE FUNCTION INTO A SINGLE OBJECTIVE FUNCTION}

$$\min \sum_{(ij)\in E}d_{ij}x_{ij} - (1-\lambda)\sum_{r \in R} y_r$$


\subsection{Median Shortest Path}

In network optimization, efficiently coordinating travel for multiple individuals to a common meeting point is a fundamental problem. The median shortest path problem (MSPP) aims to address this challenge by determining the optimal path that minimizes the total distance covered by all the participants.

(...)

The linear programming problem is formulated as follows:\\

\textbf{1. Decision variables}
\begin{itemize}
    \item $x^s_{ij}$: Binary variable indicating if student $s$ travels along edge $(i,j)$:
    
    $$x^s_{ij} = \begin{cases}
        1, & \text{if edge } (i,j) \text{ is traveled by student } s\\
        0, & \text{otherwise}
    \end{cases}$$

    \item $m_k$: Binary variable indicating if node $k$ is chosen as the meeting point:
    
    $$m_k = \begin{cases}
        1, & \text{if node } k \text{ is the meeting point}\\
        0, & \text{otherwise}
    \end{cases}$$
\end{itemize}

\textbf{2. Objective function}
\begin{itemize}
    \item Minimize total travel distance of all students

    $$\min \sum_{s \in S} \sum_{(i,j)\in E} d_{ij} x^s_{ij}$$
    \begin{itemize}
        \item $S$: The set of students
        \item $E$: The set of edges in the graph
        \item $d_{ij}$: Distance (weight) of edge $(i,j)$
    \end{itemize}
\end{itemize}

\textbf{3. Constraints}
\begin{itemize}
    \item Balance equation / Flow conservation
    $$\sum_{j \in N} x^s_{ij} - \sum_{j \in N} x^s_{ji} = \begin{cases}
        1, & \text{if } i=s\\
        -1, & \text{if } i=m\\
        0, & \text{otherwise}
    \end{cases}$$
    \begin{itemize}
        \item $s$: The starting node
        \item $m$: The meeting node
    \end{itemize}

    \item One meeting node is chosen
    $$\sum_{k \in N} m_k =1$$
    \begin{itemize}
        \item Ensures that only one meeting node is chosen
    \end{itemize}

    \item Student must arrive at meeting node
    $$\dots$$

    \item Binary conditions on decision variables
    $$x^s_{ij} \in \{0,1\}, \; \forall (i,j) \in E, \; \forall s \in S$$
    $$m_k \in \{0,1\}, \; \forall k \in N$$
\end{itemize}


\section{Methodology}
\subsection{Shortest Path Between Two Classes}
As a first step, we need to turn UBC into an undirected graph. The UBC Campus and Community Building Department provides a set of UBC’s geospatial data on \href{https://github.com/UBCGeodata/ubc-geospatial-opendata}{Github} which we used to formulate our graph. Our graph focuses on these three files from the UBC data set:   
\begin{enumerate}
    \item Road Network Data: This comes from the GeoJSON file `ubcv\_routes.geojson', which outlines all the roads in UBC. Roads are represented with single or multiple lines, each consisting of multiple points to form the line. 
    \item Building Data: This comes from CSV files `ubcv\_buildings\_simple.csv' and 'ubcv\_poi.csv' and contain details about buildings, such as their names and coordinates.
\end{enumerate}
To make the network resemble the outline of UBC, we plotted the graph where buildings and road points are represented as vertices, and the edges represent a short distance between two vertices. Buildings are connected to the three closest roads, road points are connected to form continuous lines within the same road, and each road point is also linked to the three closest roads or buildings. This ensures the graph is connected, and there is always a path from a building to another. 
\section{Results}

\section{Discussion}

\addcontentsline{toc}{section}{References}
\bibliographystyle{alpha}
\bibliography{sample}

\end{document}